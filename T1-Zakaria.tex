%\renewcommand{\thesection}{\Alph{section}.}
%\renewcommand{\thesubsection}{\arabic{subsection}.}

\documentclass[a4paper,oneside,final,notitlepage,onecolumn,12pt]{article}%
\usepackage{amsfonts}
\usepackage{amsmath}
\usepackage{amssymb}
\usepackage{boxedminipage}
\usepackage{color}
\usepackage{enumerate}
\usepackage{fancybox}
\usepackage{fancyhdr}
\usepackage{geometry}
\usepackage{graphicx}
\usepackage{lastpage}
\usepackage{multicol}
\usepackage{sectsty}
\usepackage[singlespacing]{setspace}
\usepackage[small,compact,rigidchapters]{titlesec}
\usepackage[normalem]{ulem}
\usepackage{verbatim}
\usepackage{hyperref}%
\setcounter{MaxMatrixCols}{30}
%TCIDATA{OutputFilter=latex2.dll}
%TCIDATA{Version=5.50.0.2960}
%TCIDATA{CSTFile=40 LaTeX article.cst}
%TCIDATA{Created=Wednesday, June 26, 2013 14:34:27}
%TCIDATA{LastRevised=Tuesday, August 29, 2017 18:06:46}
%TCIDATA{<META NAME="GraphicsSave" CONTENT="32">}
%TCIDATA{<META NAME="SaveForMode" CONTENT="1">}
%TCIDATA{BibliographyScheme=BibTeX}
%TCIDATA{<META NAME="DocumentShell" CONTENT="Standard LaTeX\Blank - Standard LaTeX Article">}
%TCIDATA{Language=American English}
%BeginMSIPreambleData
\providecommand{\U}[1]{\protect\rule{.1in}{.1in}}
%EndMSIPreambleData
\makeatletter
\renewcommand{\@seccntformat}[1]{}
\makeatother
\geometry{left=2.7cm,right=2cm,top=2cm,bottom=2cm}
\pagestyle{fancy}
\lhead{\color{blue}\textsc{Tugas 1}}
\chead{\color{blue}\textsc{TA-YRM}}
\rhead{\color{blue}\textsc{\footnotesize Muhammad Zakaria Musa(1103130047)}}
\lfoot{}
\cfoot{}
\rfoot{halaman \thepage\ dari \pageref{LastPage}}
\renewcommand{\headrulewidth}{0.5pt}
\renewcommand{\footrulewidth}{0.5pt}
\renewcommand{\refname}{Publications}
\sectionfont{\fontsize{12}{12}\selectfont}
\begin{document}

\begin{center}
{\large Pengenalan Diri}
\end{center}


Assalamualaikum nama saya Muhammad Zakaria Musa, saya lahir dan dibesarkan di Situbondo. Saya menjalani pendidikan menengah atas di SMAN 1 Situbondo. Pada saat itu kesibukan saya belajar dan ekstrakurikuler. Ekstrakulikuler yang saya ikuti adalah karate dan anggar. Pada tahun 2013 saya mulai mencari tahu informasi tentang universitas, dan kehidupan kulian. Pada Seleksi Nasional Masuk Perguruan Tinggi Negeri(SNMPTN) dan Seleksi Bersama Masuk Perguruan Tinggi Negeri(SBMPTN). Fakultas yang saya pilih untuk kedua seleksi tersebut adalah kedokteran Universitas Airlangga(UNAIR), dan Sekolah Teknik Elektro dan Informatika(STEI) Institut Teknologi Bandung(ITB), sayangnya saya tidak lolos pada dua seleksi tersebut. Lalu akhirnya saya mengikuti Ujian Tulis Gabungan(UTG)3 Telkom University pada tahun 2013 dengan pilihan pertama saya S1 Teknik Informatika. Alasan saya memilih Teknik Informatika awalnya karena saya salah membaca tentang prospek kerjanya. Saya salah membaca programmer dengan \textit{progamer}.

Pada tahun 2013 saya mulai berkuliah di Telkom University. Saya mengikuti program asrama yang disediakan oleh universitas. Di asrama saya banyak bertemu dengan teman-teman baru dan saya berlatih tentang kerja sama dengan tiga teman sekamar saya. Pada saat itu saya memiliki semangat untuk menjadi seseorang yang memiliki kepemimpinan yang kuat. Saya mengikuti beberapa Unit Kegiatan Mahasiswa(UKM) yaitu Gerakan \textit{Go Green}(3G), Aikido, \textit{Programming Club(Proclub)}, dan Himpunan Mahasiswa IF(HMIF). Dari UKM yang saya ikuti yang paling berkesan adalah proclub. Disana saya banyak menemukan orang-orang hebat yang menjadi idola saya. Disana juga saya bertemu dengan dean dan luke. Kami bertiga membuat tim dan mengikuti beberapa kompetisi, salah satu kompetisi yang saya kenang adalah \textit{Association for Computing Machinery International Collegiate Programming Contest}(ACM ICPC). Pada tingkat satu mata kuliah yang sulit adalah kalkulus, sedangkan yang menurut saya mudah adalah pemrograman dasar. Pada tingkat dua di semester ganjil saya melaluinya dengan baik, namun pada semester genap saya mulai lari dari permasalahan dengan bermain \textit{video game} sehingga membuat Indeks Prestasi(IP) saya jelek. Pada tingkat tiga saya melakukan hal yang sama seperti di tingkat dua. Pada semester tujuh saya mengawali perkuliahan dengan rajin namun setelah semester berakhir hanya satu mata kuliah yang lulus yaitu metode formal. Pada semester delapan saya mulai dewasa dan menghadapi permasalahan saya. Menurut saya mata kuliah yang paling relevan dengan bidang Teknik Informatika adalah mata kuliah yang berkaitan dengan paradigma pemrograman. Seperti Dasar Algoritma Pemrograman(DAP), Struktur Data(Strukdat), dan Pemrograman Berbasis Objek(PBO). Saya memilih TA metode formal karena saya merasa itu salah satu mata kuliah yang saya mengerti sekaligus saya sukai. 

Setelah lulus saya berencana melanjutkan studi, salah satu universitas yang menjadi tujuan saya adalah ITB.  Jika saya tidak dapat melanjutkan studi saya akan mencoba untuk bekerja sebagai \textit{Sofware Engineer} khusunya di bagian \textit{Back End}. Beberapa perusahan teknologi informasi yang saya tuju di Indonesia adalah Traveloka, Tokopedia, Telkom, dan Go-jek.

\end{document}